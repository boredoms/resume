% Based (but completely redesigned) on Resume in Latex Template by Sourabh Bajaj
\documentclass[letterpaper, 10pt]{article}

% font
\usepackage[T1]{fontenc}
\usepackage[sfdefault]{noto}
% \usepackage[default]{raleway}

\pdfgentounicode=1

% packages
\usepackage[empty]{fullpage}
\usepackage{titlesec}
\usepackage{scrextend}
\usepackage{hyperref}
\usepackage[dvipsnames]{xcolor}
\usepackage{fontawesome}
\usepackage{setspace}
\usepackage{enumitem}
\usepackage{pagecolor}
\usepackage{ragged2e}

\usepackage[left=1cm,right=1cm,top=1.25cm,bottom=1.25cm]{geometry}

% custom

% command macros
\newcommand{\iconSpace}{\hspace{2px}}
\newcommand{\bulletSpace}{\vspace{-4pt}}
\newcommand{\hSpace}{\hspace{8px}}
\newcommand{\secStartSpace}{\vspace{3pt}}
\newcommand{\secEndSpace}{\vspace{5pt}}
\newcommand{\spaceCollapse}{\vspace{-2pt}}

% use the macro for work experience header
% arg 1 = header title
% arg 2 = position title (subtitle (in italics)
% arg 3 = date
\newcommand{\workHeader}[3]{
\noindent \large{\textbf{\textcolor{TextColor}{#1}}} \hfill \normalsize{#3}\vspace{2pt}\\
	\textit{#2}\vspace{-2pt}
}

% use the macro for work experience subheader
\newcommand{\workSubHeader}[2]{
    \noindent \textit{#1} \hfill \normalsize{#2}
	\vspace{-2pt}
}

% use the macro for project header
\newcommand{\projectHeader}[3]{
\noindent\href{#2}{\large\textbf{#1}} \hfill \normalsize#3 \vspace{2pt}
}

% line spacing

% colors
% \definecolor{blue}{RGB}{14, 100, 237} % change the blue color if this one is not to your liking


\definecolor{PageColor}{HTML}{1b2333}
\definecolor{TextColor}{HTML}{c8d9da}
\definecolor{HeaderColor}{HTML}{eff7f8}
\definecolor{blue}{HTML}{eff7f8}
\definecolor{white}{HTML}{f0fafb}

\pagecolor{PageColor}
\color{TextColor}

\hypersetup{
    colorlinks=true,
    linkcolor=HeaderColor,
    filecolor=HeaderColor,
    urlcolor=TextColor,
}

% indent space
\titlespacing{\section}{0pt}{1px}{2ex}

% Sections formatting
\titleformat{\section}{
\vspace{-2pt}\raggedright\Large
}{}{0em}{}[\color{white}\titlerule \vspace{-5pt}]

% bolder
\newcommand{\boldlarger}[1]{{\large\textbf{\color{white}{#1}}}}
% \newcommand{\boldlarger}[1]{{\textbf{\color{white}{#1}}}}
\newcommand{\justlarge}[1]{{\large \textbf{#1}}}

\begin{document}

% ------------- header ------------- %
\begin{center} 
	{\Huge \textbf{Maximilian V\"otsch}}\\
	\vspace{1px}
 \color{TextColor}
	% "footnotes" (icon + link)
	{
		\color{white}
            \faicon{linkedin} \iconSpace \href{https://www.linkedin.com/in/Joe-Smith/}{Joe-Smith}
		\hfill
		\href{mailto:max@voets.ch}{max@voets.ch} \vspace{2pt} \iconSpace \faicon{envelope} 
        }\\
	{
            \color{white}
		\faicon{github} \iconSpace \href{https://github.com/boredoms}{boredoms}
            \hfill
		\href{https://voets.ch/}{voets.ch} \iconSpace \faicon{globe} 
		% \hSpace 
		% \faicon{map-marker} \iconSpace
		% Regina, SK
	}
\end{center}
\spaceCollapse
\spaceCollapse
% ------------- end header ------------- %


% ------------- short description ------------- %
\section{\color{blue} \textbf{Summary}}
\secStartSpace

\begin{addmargin}[0.5em]{1em}
	Outgoing PhD student interested in the design and implementation of algorithms.
	PhD work was focused on engineering algorithms for unsupervised learning problems.
	Passionate about writing efficient code and low-level optimization.
	Excellent communication skills and experienced in international collaboration.
\end{addmargin}
\secEndSpace
% ------------- end experience ------------- %


% ------------- work experience ------------- %
\section{\color{blue} \textbf{Work Experience}}
\secStartSpace

\begin{addmargin}[0.5em]{1em}
	\workHeader{University of Vienna}{Prae-Doc Assistant}{Feb 2021 - Jan 2025}
	\begin{itemize}
		\item Part of the Theory and Applications of Algorithms (TAA) group
		\item Extensive experience in teaching and mentoring students % mention concrete experience
%		\item Helped clean the floors in a team of \justlarge{3}
%		      \bulletSpace
%            \item Wrote \boldlarger{testable} and \boldlarger{well documented} code in Node.js/Nest.js backend and React frontend
	\end{itemize}
		
	% \workHeader{Examplar 2}{Founder}{May 2022 - Jan 2023}
	% \begin{itemize}
	% 	\item Created an example for users, most don't understand
		      \bulletSpace
	%	\item Created another example for users, most don't understand
	% \end{itemize}
	% \workSubHeader{Chair}{Sept 2021 - May 2022}
	% \begin{itemize}
	% 	\item Was literally a chair and productivity was up \justlarge{120\%}
	% \end{itemize}
\end{addmargin}
\secEndSpace
% ------------- end experience ------------- %



% ------------- projects ------------- %
\section{\color{blue} \textbf{Projects}}
\secStartSpace

\begin{addmargin}[0.5em]{1em}

		
	% ------------- project 1 ------------- %
	\projectHeader{XCut}{https://gitlab.com/Joe-Smith}{ 2023 - Present}
		
	\noindent A novel graph clustering algorithm implemented using random walks and the theory of expander decomposition.
	The algorithm sparsifies a graph to a tree and uses this to solve the normalized cut problem.
	\spaceCollapse
	\begin{itemize}
		\item \textbf{Technology/Tools:} C++, CMake, clang-tidy, Python, pandas, pyplot
		      \bulletSpace
		\item Source and working demo: \faicon{github} \href{https://github.com/Joe-Smith}{\underline{Joe-Smith.com/emojis-finder}}
		\item Published at KDD 2024
	\end{itemize}

     \vspace{8pt}
	% ------------- end project 1 ------------- %
 
	% ------------- project 2 ------------- %
	\projectHeader{PRONE}{https://github.com/Joe-Smith}{March 2023 - Present}
		
	\noindent A novel algorithm for solving Euclidean $k$-means or creating coresets for downstream applicaitons. 
	Made available as a Python package for the data science community.
	Main implementation in C++, with Cython wrappers provided to make it available to data scientists using python.
	\spaceCollapse
	\begin{itemize}
		\item \textbf{Technology/Tools:} C++, Python, Cython, numpy, pandas, pyplot
		      \bulletSpace
		\item Source and working demo: \faicon{github} \href{https://github.com/Joe-Smith}{\underline{Joe-Smith.com/emojis-finder}}
		\item Published at NeurIPS 2023
	\end{itemize}
	% ------------- end project 2 ------------- %
\end{addmargin}
\secEndSpace
\secEndSpace
% ------------- end projects ------------- %


% ------------- education ------------- %
\section{\color{blue} \textbf{Education}}
\secStartSpace

\begin{addmargin}[0.5em]{1em}
	\large\textbf{University of Vienna}\hfill \normalsize{Feb 2021 - (expected) March 2025}\\
	\setlength\parindent{1cm} Dr.techn. Computer Science, Thesis Title: Efficient Algorithms for Problems in Clustering and Fairness, supervised by Monika Henzinger and Kathrin Hanauer\\
	\large\textbf{University of Vienna}\hfill \normalsize{ - Aug 2021}\\
	\setlength\parindent{1cm} M.Sc. Mathematics\\
	\large\textbf{University of Vienna}\hfill \normalsize{Oct 2015 - }\\
	\setlength\parindent{1cm} B.Sc. Mathematics\\
\end{addmargin}
\secEndSpace
\secEndSpace
% ------------- end education ------------- %


% ------------- skills ------------- %
\section{\color{blue} \textbf{Skills}}
\secStartSpace

\begin{addmargin}[0.5em]{1em}
	\setstretch{1.125}
	\noindent \textbf{Languages:} C++, Python, Haskell, Rust, Java, German (native), English (fluent) \\
	\noindent \textbf{Technology:} git, unix shell, cmake, poetry \\
	\noindent \textbf{Libraries:} Blaze, OpenMP, OpenMPI, pandas, numpy, scipy-learn, pytorch \\
\end{addmargin}
\secEndSpace
\secEndSpace
% ------------- end skills ------------- %


% ------------- achievements ------------- %
\section{\color{blue} \textbf{Achievements}}
\secStartSpace

% for natHACKS specificially
\hypersetup{
    colorlinks=true,
    linkcolor=HeaderColor,
    filecolor=HeaderColor,
    urlcolor=HeaderColor,
}

\begin{addmargin}[0.5em]{1em}
    \begin{itemize}[itemsep=-2.25pt, leftmargin=1.5em]
        \item Joe Award \hfill Sept 2021 - Sept 2021
        \item Top of the Class \hfill April 2014 - April 2019
    \end{itemize}
\end{addmargin}
% ------------- end achievements ------------- %

\end{document}
