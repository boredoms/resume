% Based (but completely redesigned) on Resume in Latex Template by Sourabh Bajaj
\documentclass[a4paper, 10pt]{article}

% font
\usepackage[T1]{fontenc}
\usepackage[sfdefault]{noto}
\usepackage{ifthen}
\newboolean{darkmode}
% \usepackage[default]{raleway}

\pdfgentounicode=1

% packages
\usepackage[empty]{fullpage}
\usepackage{titlesec}
\usepackage{scrextend}
\usepackage{hyperref}
\usepackage[dvipsnames]{xcolor}
\usepackage{fontawesome}
\usepackage{setspace}
\usepackage{enumitem}
\usepackage{pagecolor}
\usepackage{ragged2e}

\usepackage[left=1cm,right=1cm,top=1.25cm,bottom=1.25cm]{geometry}

% custom

% command macros
\newcommand{\iconSpace}{\hspace{2px}}
\newcommand{\bulletSpace}{\vspace{-4pt}}
\newcommand{\hSpace}{\hspace{8px}}
\newcommand{\secStartSpace}{\vspace{3pt}}
\newcommand{\secEndSpace}{\vspace{5pt}}
\newcommand{\spaceCollapse}{\vspace{-2pt}}

% use the macro for work experience header
% arg 1 = header title
% arg 2 = position title (subtitle (in italics)
% arg 3 = date
\newcommand{\workHeader}[3]{
\noindent \large{\textbf{\textcolor{text-color}{#1}}} \hfill \normalsize{#3}\vspace{2pt}\\
	\textit{#2}\vspace{-2pt}
}

% use the macro for work experience subheader
\newcommand{\workSubHeader}[2]{
    \noindent \textit{#1} \hfill \normalsize{#2}
	\vspace{-2pt}
}

% use the macro for project header
\newcommand{\projectHeader}[3]{
\noindent\href{#2}{\large\textbf{#1}} \hfill \normalsize#3 \vspace{2pt}
}

% line spacing

% colors

% this variable defines whether to render dark mode
\setboolean{darkmode}{false}

\ifthenelse{\boolean{darkmode}}{
	\definecolor{text-color}{HTML}{c8d9da}
	\definecolor{highlight}{HTML}{eff7f8}
	\definecolor{secondary}{HTML}{f0fafb}
	\definecolor{link-color}{HTML}{eff7f8}
	\definecolor{page-color}{HTML}{1b2333}
}{
	\definecolor{text-color}{HTML}{333333}
	\definecolor{highlight}{HTML}{2c446f}
	\definecolor{secondary}{HTML}{2c446f}
	\definecolor{link-color}{HTML}{eff7f8}
	\definecolor{page-color}{HTML}{ffffff}
}

\pagecolor{page-color}
\color{text-color}

\hypersetup{
    colorlinks=true,
    linkcolor=link-color,
    filecolor=highlight,
    urlcolor=text-color,
}

% indent space
\titlespacing{\section}{0pt}{1px}{2ex}

% Sections formatting
\titleformat{\section}{
\vspace{-2pt}\raggedright\Large
}{}{0em}{}[\color{secondary}\titlerule \vspace{-5pt}]

% bolder
\newcommand{\boldlarger}[1]{{\large\textbf{\color{secondary}{#1}}}}
% \newcommand{\boldlarger}[1]{{\textbf{\color{white}{#1}}}}
\newcommand{\justlarge}[1]{{\large \textbf{#1}}}

\begin{document}

% ------------- header ------------- %
\begin{center} 
	{\Huge \color{highlight} \textbf{Maximilian V\"otsch}}\\
	\vspace{1px}
 \color{text-color}
	% "footnotes" (icon + link)
	{
		\color{secondary}
            \faicon{linkedin} \iconSpace \href{https://www.linkedin.com/in/maximilian-vötsch/}{maximilian-vötsch}
		\hfill
		\href{mailto:max@voets.ch}{max@voets.ch} \vspace{2pt} \iconSpace \faicon{envelope} 
        }\\
	{
        \color{secondary}
			\faicon{github} \iconSpace \href{https://github.com/boredoms}{boredoms}
        \hfill
		\href{https://voets.ch/}{voets.ch} \vspace{2pt} \iconSpace \faicon{globe} 
		% \hSpace 
		% \faicon{map-marker} \iconSpace
		% Regina, SK
		}\\
	{
		\color{secondary}
			\faicon{gitlab} \iconSpace \href{https://gitlab.com/voetschm}{voetschm}
		\hfill
		Vienna, Austria \vspace{2pt} \iconSpace \faicon{map-marker}
	}
\end{center}
\spaceCollapse
\spaceCollapse
% ------------- end header ------------- %


% ------------- short description ------------- %
\section{\color{highlight} \textbf{Summary}}
\secStartSpace
\begin{addmargin}[0.5em]{1em}
	Outgoing PhD student interested in the design and implementation of algorithms.
	My PhD work was focused on engineering algorithms for unsupervised learning problems.
	I'm passionate about writing efficient code and low-level optimization.
	I have excellent communication skills and am experienced in international collaboration.
\end{addmargin}
\secEndSpace
\secEndSpace
% ------------- end experience ------------- %


% ------------- work experience ------------- %
\section{\color{highlight} \textbf{Work Experience}}
\secStartSpace
\begin{addmargin}[0.5em]{1em}
	\workHeader{University of Vienna}{Prae-Doc Assistant}{Feb 2021 - Jan 2025}
	\begin{itemize}
		\item Part of the Theory and Applications of Algorithms (TAA) group, where I used methods of classical algorithms research to design algorithms for unsupervised learning objectives and implemented them in C++
		\item Extensive experience in supervising and teaching students in mathematics, algorithms, data structures % mention concrete experience
		\item Experience in oganizing projects, conferences and workshops
%		\item Helped clean the floors in a team of \justlarge{3}
%		      \bulletSpace
%            \item Wrote \boldlarger{testable} and \boldlarger{well documented} code in Node.js/Nest.js backend and React frontend
	\end{itemize}
	% \workHeader{Examplar 2}{Founder}{May 2022 - Jan 2023}
	% \begin{itemize}
	% 	\item Created an example for users, most don't understand
	%	      \bulletSpace
	%	\item Created another example for users, most don't understand
	% \end{itemize}
	% \workSubHeader{Chair}{Sept 2021 - May 2022}
	% \begin{itemize}
	% 	\item Was literally a chair and productivity was up \justlarge{120\%}
	% \end{itemize}
\end{addmargin}
\secEndSpace
% ------------- end experience ------------- %



% ------------- projects ------------- %
\section{\color{highlight} \textbf{Projects}}
\secStartSpace

\begin{addmargin}[0.5em]{1em}

		
	% ------------- project 1 ------------- %
	\projectHeader{XCut (published at KDD 2024)}{https://gitlab.com/vietaa/xcut}{May 2023 - Present}
		
	\noindent A graph clustering algorithm implemented using random walks and the theory of expander decomposition.
	The algorithm sparsifies a graph to a tree and uses this to solve the normalized cut problem. Current state of the art solver for this problem.
	\spaceCollapse
	\begin{itemize}
		\item \textbf{Technologies:} C++, CMake, Python, pandas, pyplot
		      \bulletSpace
	\end{itemize}

     \vspace{8pt}
	% ------------- end project 1 ------------- %
 
	% ------------- project 2 ------------- %
	\projectHeader{PRONE (published at NeurIPS 2023)}{https://github.com/boredoms/prone}{February 2023 - Present}
		
	\noindent New algorithm for solving Euclidean $k$-means and creating coresets for downstream applications in time $O(n \log n)$. 
	Made available as a Python package for the data science community.
	Main implementation in C++, with Cython wrappers provided to make it available to data scientists using python.
	\spaceCollapse
	\begin{itemize}
		\item \textbf{Technologies:} C++, Python, Cython, numpy, pandas, pyplot
		      \bulletSpace
	\end{itemize}
	% ------------- end project 2 ------------- %
\end{addmargin}
\secEndSpace
\secEndSpace
% ------------- end projects ------------- %


% ------------- education ------------- %
\section{\color{highlight} \textbf{Education}}
\secStartSpace

\begin{addmargin}[0.5em]{1em}
	\large\textbf{University of Vienna}\hfill \normalsize{February 2021 - (expected) March 2025}\\
	\setlength\parindent{1cm} Dr. techn. Computer Science\\
	Supervisors: Monika Henzinger and Kathrin Hanauer\\
	\vspace{8pt}\hspace{-2pt}
	Thesis Title: Efficient Algorithms for Problems in Clustering and Fairness\\
	\large\textbf{University of Vienna}\hfill \normalsize{October 2014 - August 2020}\\
	\setlength\parindent{1cm} B.Sc. and M.Sc. Mathematics\\
%	\large\textbf{University of Vienna}\hfill \normalsize{October 2014 - March 2018}\\
%	 \setlength\parindent{1cm} B.Sc. Mathematics\\
\end{addmargin}
\secEndSpace
\secEndSpace
% ------------- end education ------------- %


% ------------- skills ------------- %
\section{\color{highlight} \textbf{Skills}}
\secStartSpace

\begin{addmargin}[0.5em]{1em}
	\setstretch{1.125}
	\noindent \textbf{Languages:} C++, Python, Haskell, Rust, Java, German (native), English (fluent) \\
	\noindent \textbf{Technology:} Linux, git, unix shell, cmake, poetry, clang-tidy \\
	\noindent \textbf{Libraries:} Blaze, OpenMP, OpenMPI, pandas, numpy, scikit-learn, pytorch \\
\end{addmargin}
\secEndSpace
\secEndSpace
% ------------- end skills ------------- %


% ------------- achievements ------------- %
\section{\color{highlight} \textbf{Awards}}
\secStartSpace

\begin{addmargin}[0.5em]{1em}
    \begin{itemize}[itemsep=-2.25pt, leftmargin=1.5em]
		\item KDD 2024 Audience Appreciation Award \hfill August 2024
    \end{itemize}
\end{addmargin}
\secEndSpace
\secEndSpace
% ------------- end achievements ------------- %

% ------------- personal ------------- %
\section{\color{highlight} \textbf{Personal Interests}}
\secStartSpace
\begin{addmargin}[0.5em]{1em}
	Bouldering, Analog and Digital Photography, Electronics, Guitar
\end{addmargin}
% ------------- end personal ------------- %


\end{document}
