% Based (but completely redesigned) on Resume in Latex Template by Sourabh Bajaj
\documentclass[a4paper, 10pt]{article}

% language setup
\usepackage[main=english,ngerman,japanese]{babel}

% font
\usepackage[T1]{fontenc}
\usepackage[sfdefault]{noto}
\usepackage{xifthen}
% \usepackage[default]{raleway}

% this is for using a japanese language font, if you want to use another language that does not use latin characters, add the fonts here in a similar way
\usepackage{luatexja-fontspec}
\usepackage[noto-otf,fontspec,match]{luatexja-preset}

% \pdfgentounicode=1

% packages
\usepackage[empty]{fullpage}
\usepackage{titlesec}
\usepackage{scrextend}
\usepackage{hyperref}
\usepackage[dvipsnames]{xcolor}
\usepackage{fontawesome}
\usepackage{setspace}
\usepackage{enumitem}
\usepackage{pagecolor}
\usepackage{ragged2e}

\usepackage[left=1cm,right=1cm,top=1.25cm,bottom=1.25cm]{geometry}

% custom command macros

% helpers
\newcommand{\ifthen}[2]{\ifthenelse{#1}{#2}{}}
% bolder
\newcommand{\boldlarger}[1]{{\large\textbf{\color{secondary}{#1}}}}
% \newcommand{\boldlarger}[1]{{\textbf{\color{white}{#1}}}}
\newcommand{\justlarge}[1]{{\large \textbf{#1}}}

% formatting macros

% setting up the formatting for the section
% Sections formatting
\titlespacing{\section}{0pt}{1px}{2ex}
\titleformat{\section}{
\vspace{-2pt}\raggedright\Large
}{}{0em}{}[\color{secondary}\titlerule \vspace{-5pt}]

% macro for a section of the resume
\newcommand{\resumeSection}[2]{
\section{\color{highlight}\textbf{#1}}
\secStartSpace
\begin{addmargin}[0.5em]{1em}
	#2
\end{addmargin}
\secEndSpace
}

% macro for work experience header
% arg 1 = header title
% arg 2 = position title (subtitle (in italics)
% arg 3 = date
\newcommand{\workHeader}[3]{
\noindent\large{\textbf{\textcolor{text-color}{#1}}}\hfill\normalsize{#3}\vspace{2pt}\\
	\textit{#2}\vspace{-2pt}
}

% macro for work experience subheader
\newcommand{\workSubHeader}[2]{
    \noindent\textit{#1}\hfill\normalsize{#2}
	\vspace{-2pt}
}

% macro for an item in the work-section. it uses the work-header macro to create an entry for the section
\newcommand{\workItem}[4]{
	\workHeader{#1}{#2}{#3}
	\noindent #4
}

% similar macros for the other sections follow

% macro for the project header
\newcommand{\projectHeader}[4]{
\noindent\href{#2}{\large\textbf{\textcolor{text-color}{#1}}}\hfill\normalsize{#3}\vspace{2pt}\\
	\textit{#4}\vspace{2pt}
}

\newcommand{\projectItem}[5]{
	\projectHeader{#1}{#2}{#3}{#4}\\
	\noindent #5
}

\newcommand{\educationItem}[3]{
	\noindent\large\textbf{#1}\hfill\normalsize{#2}\\
	#3 \vspace{8pt}\\
}

\newcommand{\educationItemLast}[3]{
	\noindent\large\textbf{#1}\hfill\normalsize{#2}\\
	#3
}

\newcommand{\skillItem}[2]{
	\noindent\textbf{#1}\hfill #2
}

\newcommand{\awardItem}[2]{
	\noindent\textbf{#1}\hfill #2
}

% spacing
\newcommand{\iconSpace}{\hspace{2px}}
\newcommand{\bulletSpace}{\vspace{-4pt}}
\newcommand{\hSpace}{\hspace{8px}}
\newcommand{\secStartSpace}{\vspace{3pt}}
\newcommand{\secEndSpace}{\vspace{5pt}}
\newcommand{\spaceCollapse}{\vspace{-2pt}}
\newcommand{\skillSpacing}{\setstretch{1.125}}

% text colors
\newcommand{\darkmodeColors}{
	\definecolor{text-color}{HTML}{c8d9da}
	\definecolor{highlight}{HTML}{eff7f8}
	\definecolor{secondary}{HTML}{f0fafb}
	\definecolor{link-color}{HTML}{eff7f8}
	\definecolor{page-color}{HTML}{1b2333}
}
\newcommand{\lightmodeColors}{
	\definecolor{text-color}{HTML}{333333}
	\definecolor{highlight}{HTML}{2c446f}
	\definecolor{secondary}{HTML}{2c446f}
	\definecolor{link-color}{HTML}{eff7f8}
	\definecolor{page-color}{HTML}{ffffff}
}

\newcommand{\configureColors}{
\pagecolor{page-color}
\color{text-color}
\hypersetup{
    colorlinks=true,
    linkcolor=link-color,
    filecolor=highlight,
    urlcolor=text-color,
}
}


% commands for bilingual document

% define a boolean based on the variable set in the -en.tex file
\newboolean{xen}
\ifdefined\isEnglish
	\setboolean{xen}{true}
\fi

% same for german
\newboolean{xde}
\ifdefined\isGerman
	\setboolean{xde}{true}
\fi

% commands to print a language block, uses otherlanguage environment to 
% correctly enforce line-breaking and other stylistic things
\newcommand{\en}[1]{\ifthen{\boolean{xen}}{#1}}
\newcommand{\de}[1]{\ifthen{\boolean{xde}}{
	\begin{otherlanguage}{ngerman}
	#1
	\end{otherlanguage}
}}

% case-switch implementation - from Thev (https://tex.stackexchange.com/questions/64131/implementing-switch-cases)
\newcommand{\ifequals}[3]{\ifthenelse{\equal{#1}{#2}}{#3}{}}
\newcommand{\case}[2]{#1 #2} % Dummy, so \renewcommand has something to overwrite...
\newenvironment{switch}[1]{\renewcommand{\case}{\ifequals{#1}}}{}

% define feature booleans
\newboolean{darkmode}
\newboolean{summary}
\newboolean{awards}
\newboolean{personal}

% choose which header style (
% % 1 - 3 lines
% % 2 - 2 lines compact
% % 3 - 2 lines, no location/website 
\newcommand{\headerStyle}{2}
% enable dark mode
\setboolean{darkmode}{false}
% include executive summary
\setboolean{summary}{false}
% include awards in their own section
\setboolean{awards}{false}
% include "personal" section
\setboolean{personal}{true}

% configure the colors
\ifthenelse{\boolean{darkmode}}{
	\darkmodeColors
}{
	\lightmodeColors
}
\configureColors

\begin{document}
% ------------- header ------------- %
% TODO: clean up the header and separate design and content
% Name
\begin{center} 
	{\Huge \color{highlight} \textbf{Maximilian V\"otsch}}\\
	\vspace{1px}

% Icon header style
\begin{switch}{\headerStyle}
	\case{1}{
 	{
		\color{secondary}
            \faicon{linkedin} \iconSpace \href{https://www.linkedin.com/in/maximilian-vötsch/}{maximilian-vötsch}
		\hfill
		\href{mailto:max@voets.ch}{max@voets.ch} \vspace{2pt} \iconSpace \faicon{envelope} 
    }\\
	{
        \color{secondary}
			\faicon{github} \iconSpace \href{https://github.com/boredoms}{boredoms}
        \hfill
		\href{https://voets.ch/}{voets.ch} \vspace{2pt} \iconSpace \faicon{globe} 
	}\\
	{
		\color{secondary}
			\faicon{gitlab} \iconSpace \href{https://gitlab.com/voetschm}{voetschm}
		\hfill
		\href{https://voets.ch}{Vienna, Austria} \vspace{2pt}\iconSpace\faicon{map-marker} 
	}
	}


	\case{2}{
		{ % own group, so we can change the spacing
		\vspace{0.25em}
        \color{secondary}
		\setlength{\tabcolsep}{0.2em}
		\begin{tabular}{cl@{\hskip 2em}cl@{\hskip 2em}cl}
            \faicon{linkedin}  & \href{https://www.linkedin.com/in/maximilian-vötsch/}{maximilian-vötsch} &
			\faicon{github} & \href{https://github.com/boredoms}{boredoms} & 
			\faicon{gitlab} & \href{https://gitlab.com/voetschm}{voetschm} \\
			\faicon{envelope} & \href{mailto:max@voets.ch}{max@voets.ch} & 
			\faicon{globe} & \href{https://voets.ch/}{voets.ch} & 
			\faicon{map-marker} & \href{https://voets.ch}{Vienna, Austria} \\
		\end{tabular}
		}
	}

	\case{3}{
 	{
		\color{secondary}
            \faicon{linkedin} \iconSpace \href{https://www.linkedin.com/in/maximilian-vötsch/}{maximilian-vötsch}
		\hfill
		\href{mailto:max@voets.ch}{max@voets.ch} \vspace{2pt} \iconSpace \faicon{envelope} 
    }\\
	{
        \color{secondary}
			\faicon{github} \iconSpace \href{https://github.com/boredoms}{boredoms}
        \hfill
		\href{https://gitlab.com/voetschm}{voetschm} \vspace{2pt} \iconSpace \faicon{gitlab} 
	}\\
	}
\end{switch}
\end{center}
\spaceCollapse
\spaceCollapse

% ------------- end header ------------- %


% ------------- short description ------------- %
\ifthen{\boolean{summary}}{
\en{\resumeSection{Summary}{
	Outgoing PhD student interested in the design and implementation of algorithms.
	My PhD work was focused on engineering algorithms for unsupervised learning.
	I am passionate about writing efficient code and low-level optimization.
	Throughout my studies I also gained significant experience in managing projects and international collaboration.
}}
\de{\resumeSection{Zusammenfassung}{
	% TODO
}}
}
% ------------- end experience ------------- %


% ------------- work experience ------------- %
\en{
\resumeSection{Work Experience}{
	\workItem{University of Vienna}{Prae-Doc Assistant in the Theory and Applications of Algorithms (TAA) group}{Feb 2021 - Ongoing}{
	\begin{itemize}
		\setlength{\itemsep}{0em}
		\item Researched how tools from classical algorithm research can be used to design efficient algorithms for unsupervised learning objectives 
		\item Implemented, benchmarked, and optimized algorithms in C++ following algorithm engineering practices
		\item Organized workshops and conferences, co-organizing the Queer in AI workshop at ICML 2024 and local organizer of the SEA 2024 conference
		\item Collaborated internationally with researchers from academia and industry (Stanford, CMU, TU Munich, IIT Delhi, Google, ...)
		\item Acted as an expert reviewer for high profile conferences (NeurIPS, KDD, ICML, ALENEX, ICALP, SEA, ...)
		\item Co-supervised a Bachelor's thesis %on "Graph Clustering: A Comparison of Louvain and Leiden" 
and a Masters thesis. % on the topic "Repetition Free Longest Common Subsequence".
Taught the courses "Advanced Algorithms" and "Algorithms and Data Structures for Computational Science". Taught the exercise class for "Mathematical Foundations of Computer Science 1" for six semesters. % Assisted in teaching "Algorithms and Data Structures 2" once.
		\ifthen{\not\boolean{awards}}{\item{Received the faculty award for significant contributions in the category Publications in Highest Ranking Venues in 2023}}
%		\item Helped clean the floors in a team of \justlarge{3}
%		      \bulletSpace
%            \item Wrote \boldlarger{testable} and \boldlarger{well documented} code in Node.js/Nest.js backend and React frontend
	% \workHeader{Examplar 2}{Founder}{May 2022 - Jan 2023}
	% \begin{itemize}
	% 	\item Created an example for users, most don't understand
	%	      \bulletSpace
	%	\item Created another example for users, most don't understand
	% \end{itemize}
	% \workSubHeader{Chair}{Sept 2021 - May 2022}
	% \begin{itemize}
	% 	\item Was literally a chair and productivity was up \justlarge{120\%}
	% \end{itemize}
	\end{itemize}
	}
}
}
\de{
\resumeSection{Arbeitserfahrung}{
	\workItem{Universit\"at Wien}{Prae-Doc Assistent in der Forschungsgruppe Theory and Applications of Algorithms (TAA)}{Feb 2021 - Laufend}{
	\begin{itemize}
		\setlength{\itemsep}{0em}
		\item Forschung dazu wie mit Methoden aus der Algorithmentheorie effizientere Algorithmen f\"ur unsupervised Learning entwickelt werden k\"onnen
		\item Implementierung, Benchmarking und Optimierung von Algorithmen in C++ nach Algorithm Engineering Praktiken
		\item Organisationserfahrung bei Workshops und Konferenzen, z.B. Organisator des Queer in AI Workshop bei ICML 2024 und lokaler Organisator der SEA 2024 Konferenz
		\item Erfahrung mit internationaler Kollaboration, sowohl Akademisch (Stanford, CMU, TU M\"unchen, IIT Delhi, ...) als auch Industrie (Google)
		\item Expert Reviewer f\"ur hochrangige Konferenzen (NeurIPS, KDD, ICML, ALENEX, ICALP, SEA, ...)
		\item Mitbetreuung von Bachelorstudenten und Masterstudenten.%(Thema: Graph Clustering: A Comparison of Louvain and Leiden) und Masterstudenten (Thema: Repetition Free Longest Common Subsequence). 
Unterrichten der Kurse ``Advanced Algorithms'' und ``Algorithms and Data Structures for Computational Science'', sowie der ``PUE Mathematical Foundations of Computer Science 1''. %Teaching Assistant f\"ur den Kurs ``Algorithms and Data Structures 2''.
		\ifthen{\not\boolean{awards}}{\item{Erhalt des Fakult\"atsawards f\"ur signifikante Beitr\"age in der Kategorie Publikationen in h\"ochstrangigen Venues f\"ur 2023}}
	\end{itemize}
	}
}
}
% ------------- end experience ------------- %



% ------------- projects ------------- %
\en{
\resumeSection{Projects}{
	\projectItem{XCut (published at KDD 2024)}{https://gitlab.com/vietaa/xcut}{May 2023 - Ongoing}{First practical algorithm using expander decomposition to cluster a graph.}{XCut solves the normalized cut problem by sparsifying a graph to a tree and it is the current state of the art solver for this problem.
		I co-designed the algorithm, implemented it in \emph{C++}, and performed all experiments and data analysis using \emph{Python}. 
		\ifthen{\not\boolean{awards}}{This project received the audience appreciation award at KDD 2024, which is awarded to papers that garner significant public interest.}
 	   \vspace{8pt}
	}

	\projectItem{PRONE (published at NeurIPS 2023)}{https://github.com/boredoms/prone}{February 2023 - Ongoing}{New algorithm for solving Euclidean $k$-means and creating coresets for downstream applications.}{The algorithm has a running time of $O(nnz(A) + n \log n)$. 
		It has been made available as a \emph{Python} package, with its main implementation in \emph{C++}, and \emph{Cython} wrappers provided to make it available to data scientists.
		I co-designed the algorithm, implemented it for preliminary experiments, analyzed and plotted data using \emph{pandas} and \emph{pyplot}, and provided the resulting algorithm as a Python package.
	}
}
}
\de{
\resumeSection{Projekte}{
	\projectItem{XCut (publiziert bei KDD 2024)}{https://gitlab.com/vietaa/xcut}{Mai 2023 - Laufend}{Der erste Algorithmus f\"ur Graph-Clustering der auf Expanderzerlegung basiert.}{XCut l\"ost das Normalized Cut Problem auf Graphen durch Sparsifizierung des Graphen zu einem Baum und ist der derzeitige State-of-the-Art Solver f\"ur dieses Problem.
		Ich habe am Design des Algorithmus gearbeitet, ihn in \emph{C++} implementiert und alle Experimente, sowie die Datenanalyse in Python durchgef\"uhrt.
		\ifthen{\not\boolean{awards}}{Das Projekt wurde mit dem Audience Appreciation Award der KDD 2024 geehrt, welcher an Paper mit hohem \"offentlichen Interesse geht.}
		\vspace{8pt}

	}
	\projectItem{PRONE (publizert bei NeurIPS 2023)}{https://github.com/boredoms/prone}{Februar 2023 - Laufend}{Ein neuer Algorithmus zum l\"osen des Euclidean $k$-means Problems und zum Erstellen von Coresets.}{Die Laufzeit des Algorithmus ist $O(nnz(A) + n \log n)$. 
		Der Algorithmus ist als Python Package f\"ur Data Scientists verf\"ugbar.
		Die Hautpimplementation des Algorithmus ist in \emph{C++}, mit \emph{Cython} Wrapper, um performante \emph{Python} bindings zur Verf\"ugung stellen zu k\"onnen.
		Ich habe den Algorithmus co-designed, ihn in \emph{C++} implementiert und alle Experimente und die Datenaufarbeitung durchgef\"uhrt.
	}
}
}
% ------------- end projects ------------- %


% ------------- education ------------- %
\en{
\resumeSection{Education}{
	\educationItem{University of Vienna}{February 2021 - Ongoing}{
	Dr. techn. Computer Science\\
	Supervisors: Univ.-Prof. Dr. Monika Henzinger and Ass.-Prof. Dr. Kathrin Hanauer, B.Sc. M.SC.\\
	Thesis Title: Efficient Algorithms for Problems in Clustering and Fairness
	}
	\educationItem{University of Vienna}{March 2018 - August 2020}{
    M.Sc. Mathematics, Thesis title: Cofinitary Groups
	}
	\educationItem{University of Vienna}{October 2016 - January 2021}{
	B.Sc. Computer Science, did not graduate
	}
	\educationItemLast{University of Vienna}{October 2014 - March 2018}{
	B.Sc. Mathematics, Thesis title: Lattice Path Matroids
	}
}
}
\de{
\resumeSection{Ausbildung}{
	\educationItem{Universit\"at Wien}{Februar 2021 - Laufend}{
	Dr. techn. Informatik\\
	Betreuer: Univ.-Prof. Dr. Monika Henzinger und Ass.-Prof. Dr. Kathrin Hanauer, B.Sc. M.SC.\\
	Thema: Efficient Algorithms for Problems in Clustering and Fairness
	}
	\educationItem{Universit\"at Wien}{M\"arz 2018 - August 2020}{
	M.Sc. Mathematik, Thema der Abschlussarbeit: Cofinitary Groups
	}
	\educationItem{Universit\"at Wien}{Oktober 2016 - Januar 2021}{
	B.Sc. Informatik, nicht Abgeschlossen
	}
	\educationItemLast{Universit\"at Wien}{Oktober 2014 - M\"arz 2018}{
	B.Sc. Mathematik, Thema der Abschlussarbeit: Lattice Path Matroids
	}
}
}
% ------------- end education ------------- %


% ------------- skills ------------- %
\en{
\resumeSection{Skills}{
	\skillSpacing
	\skillItem{Languages}{C++, Python, Haskell, Rust, German (native), English (fluent)} \\
	\skillItem{Technologies}{Linux, git, shell, cmake, uv, vim, Docker, GitLab CI} \\
	\skillItem{Libraries}{Blaze, OpenMP, OpenMPI, Catch2, pandas, numpy, scikit-learn, pytorch}
}
}
\de{
\resumeSection{F\"ahigkeiten}{
	\skillSpacing
	\skillItem{Sprachen}{C++, Python, Haskell, Rust, German (native), English (fluent)} \\
	\skillItem{Technologien}{Linux, git, unix shell, cmake, uv, vim, Docker, GitLab CI} \\
	\skillItem{Libraries}{Blaze, OpenMP, OpenMPI, Catch2, pandas, numpy, scikit-learn, pytorch}
}
}
% ------------- end skills ------------- %


% ------------- achievements ------------- %
\ifthen{\boolean{awards}}{
\en{
\resumeSection{Awards}{
	\awardItem{Faculty Award for Significant Contributions in the Category Publications in Highest Ranking Venues in 2023}{December 2024}\\
	\awardItem{KDD 2024 Audience Appreciation Award}{August 2024}
}
}
\de{
\resumeSection{Auszeichnungen}{
	\awardItem{Faculty Award for Significant Contributions in the Category Publications in Highest Ranking Venues in 2023}{Dezember 2024}\\
	\awardItem{KDD 2024 Audience Appreciation Award}{August 2024}
}
}
}
% ------------- end achievements ------------- %

% ------------- personal ------------- %
\ifthen{\boolean{personal}}{
	\en{
	\resumeSection{Personal Interests}{
		Bouldering, Analog and Digital Photography, Electronics, Guitar
	}
	}
	\de{
	\resumeSection{Pers\"onliche Interessen}{
		Bouldern, Analog- und Digitalphotographie, Mikroelektronik, Gitarre
	}
	}
	\jp{
	\resumeSection{私益}{
		 ボルダリング, 撮影, 
	}
	}
}
% ------------- end personal ------------- %


\end{document}
