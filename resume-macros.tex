% custom command macros

% helpers
\newcommand{\ifthen}[2]{\ifthenelse{#1}{#2}{}}
% bolder
\newcommand{\boldlarger}[1]{{\large\textbf{\color{secondary}{#1}}}}
% \newcommand{\boldlarger}[1]{{\textbf{\color{white}{#1}}}}
\newcommand{\justlarge}[1]{{\large \textbf{#1}}}

% formatting macros

% setting up the formatting for the section
% Sections formatting
\titlespacing{\section}{0pt}{1px}{2ex}
\titleformat{\section}{
\vspace{-2pt}\raggedright\Large
}{}{0em}{}[\color{secondary}\titlerule \vspace{-5pt}]

% macro for a section of the resume
\newcommand{\resumeSection}[2]{
\section{\color{highlight}\textbf{#1}}
\secStartSpace
\begin{addmargin}[0.5em]{1em}
	#2
\end{addmargin}
\secEndSpace
}

% macro for work experience header
% arg 1 = header title
% arg 2 = position title (subtitle (in italics)
% arg 3 = date
\newcommand{\workHeader}[3]{
\noindent\large{\textbf{\textcolor{text-color}{#1}}}\hfill\normalsize{#3}\vspace{2pt}\\
	\textit{#2}\vspace{-2pt}
}

% macro for work experience subheader
\newcommand{\workSubHeader}[2]{
    \noindent\textit{#1}\hfill\normalsize{#2}
	\vspace{-2pt}
}

% macro for an item in the work-section. it uses the work-header macro to create an entry for the section
\newcommand{\workItem}[4]{
	\workHeader{#1}{#2}{#3}
	\noindent #4
}

% similar macros for the other sections follow

% macro for the project header
\newcommand{\projectHeader}[4]{
\noindent\href{#2}{\large\textbf{\textcolor{text-color}{#1}}}\hfill\normalsize{#3}\vspace{2pt}\\
	\textit{#4}\vspace{2pt}
}

\newcommand{\projectItem}[5]{
	\projectHeader{#1}{#2}{#3}{#4}\\
	\noindent #5
}

\newcommand{\educationItem}[3]{
	\noindent\large\textbf{#1}\hfill\normalsize{#2}\\
	#3 \vspace{8pt}\\
}

\newcommand{\educationItemLast}[3]{
	\noindent\large\textbf{#1}\hfill\normalsize{#2}\\
	#3
}

\newcommand{\skillItem}[2]{
	\noindent\textbf{#1}\hfill #2
}

\newcommand{\awardItem}[2]{
	\noindent\textbf{#1}\hfill #2
}

% spacing
\newcommand{\iconSpace}{\hspace{2px}}
\newcommand{\bulletSpace}{\vspace{-4pt}}
\newcommand{\hSpace}{\hspace{8px}}
\newcommand{\secStartSpace}{\vspace{3pt}}
\newcommand{\secEndSpace}{\vspace{5pt}}
\newcommand{\spaceCollapse}{\vspace{-2pt}}
\newcommand{\skillSpacing}{\setstretch{1.125}}

% text colors
\newcommand{\darkmodeColors}{
	\definecolor{text-color}{HTML}{c8d9da}
	\definecolor{highlight}{HTML}{eff7f8}
	\definecolor{secondary}{HTML}{f0fafb}
	\definecolor{link-color}{HTML}{eff7f8}
	\definecolor{page-color}{HTML}{1b2333}
}
\newcommand{\lightmodeColors}{
	\definecolor{text-color}{HTML}{333333}
	\definecolor{highlight}{HTML}{2c446f}
	\definecolor{secondary}{HTML}{2c446f}
	\definecolor{link-color}{HTML}{eff7f8}
	\definecolor{page-color}{HTML}{ffffff}
}

\newcommand{\configureColors}{
\pagecolor{page-color}
\color{text-color}
\hypersetup{
    colorlinks=true,
    linkcolor=link-color,
    filecolor=highlight,
    urlcolor=text-color,
}
}


% commands for bilingual document

% define a boolean based on the variable set in the -en.tex file
\newboolean{xen}
\ifdefined\isEnglish
	\setboolean{xen}{true}
\fi

% same for german
\newboolean{xde}
\ifdefined\isGerman
	\setboolean{xde}{true}
\fi

% commands to print a language block, uses otherlanguage environment to 
% correctly enforce line-breaking and other stylistic things
\newcommand{\en}[1]{\ifthen{\boolean{xen}}{#1}}
\newcommand{\de}[1]{\ifthen{\boolean{xde}}{
	\begin{otherlanguage}{ngerman}
	#1
	\end{otherlanguage}
}}

% case-switch implementation - from Thev (https://tex.stackexchange.com/questions/64131/implementing-switch-cases)
\newcommand{\ifequals}[3]{\ifthenelse{\equal{#1}{#2}}{#3}{}}
\newcommand{\case}[2]{#1 #2} % Dummy, so \renewcommand has something to overwrite...
\newenvironment{switch}[1]{\renewcommand{\case}{\ifequals{#1}}}{}

% define feature booleans
\newboolean{darkmode}
\newboolean{summary}
\newboolean{awards}
\newboolean{personal}